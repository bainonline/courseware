% Created 2024-07-03 Wed 20:41
% Intended LaTeX compiler: pdflatex
\documentclass[11pt]{article}
\usepackage[utf8]{inputenc}
\usepackage[T1]{fontenc}
\usepackage{graphicx}
\usepackage{longtable}
\usepackage{wrapfig}
\usepackage{rotating}
\usepackage[normalem]{ulem}
\usepackage{amsmath}
\usepackage{amssymb}
\usepackage{capt-of}
\usepackage{hyperref}
\author{Abhijit Bhopatkar}
\date{\today}
\title{Igcse Physics 0625 0972}
\hypersetup{
 pdfauthor={Abhijit Bhopatkar},
 pdftitle={Igcse Physics 0625 0972},
 pdfkeywords={},
 pdfsubject={},
 pdfcreator={Emacs 29.4 (Org mode 9.6.24)}, 
 pdflang={English}}
\begin{document}

\maketitle
\tableofcontents

\begin{center}
\begin{tabular}{lrlll}
\hline
Month & Week & Dates & Sessions & Plan\\[0pt]
\hline
June & 1 & \textit{<2024-06-24 Mon>} & Introduction & Units and measurements\\[0pt]
 &  & \textit{<2024-06-25 Tue>} &  & Lengths\\[0pt]
 &  &  &  & Weights\\[0pt]
 &  &  &  & Volume\\[0pt]
 &  &  &  & Density\\[0pt]
 &  &  &  & Length\\[0pt]
 &  &  & Measurements & Repeatability\\[0pt]
 &  &  &  & Accuracy\\[0pt]
 &  &  &  & Errors\\[0pt]
\hline
July & 2 & \textit{<2024-07-01 Mon>} & Measurements & Measuring time\\[0pt]
 &  & \textit{<2024-07-02 Tue>} &  & Estimation\\[0pt]
 &  &  &  & Practical examples\\[0pt]
\hline
 &  & TBD & Practical & Measure Earth's circumference\\[0pt]
 &  &  &  & Measure Density of a human\\[0pt]
\hline
 &  & TBD & Recall & Practice problems and quiz\\[0pt]
\hline
\end{tabular}
\end{center}


\section{Physics}
\label{sec:org17f8b96}
\subsection{Units, Measurements and Accuracy}
\label{sec:orgbcff223}
The aim of this unit is to describe different methods and aspects of physical measurements.
Introduction to inaccuracy and errors. Types units and recall of all the things they learned about lengths, weights, volumes
Introduction to density.

Practical measurements.


\begin{enumerate}
\item What are units?
\item How is length measured?
\item Accuracy or error in measurements.
\item Measuring earth's circumference.
\item Volumes.
\item Mass.
\item Density.
\item Measuring density of a human being.
\end{enumerate}

\subsubsection{Experiments}
\label{sec:org68e4647}
\begin{enumerate}
\item Measure a height of a building.
Measure length of the shadow of the building.
Measure length of a shadow of a meter stick.
\end{enumerate}
\begin{verbatim}

      +--=
      |  |=
     a|  | =       |-
      |  |  =     b| -
      +--=====     |===
           c         d

\end{verbatim}
\[
   \frac{a}{b} = \frac{c}{d} \therefore a = \frac{b * c}{d}
   \]

\begin{enumerate}
\item Measure circumference|of the earth.
Find which city has zero shadow day today and measure a shadow of long stick.
Use Eratosthenes's method to calculate circumference of Earth.

\item Measuring density of a human.
Dunk a child in a drum of water?
\end{enumerate}




\subsection{Lecture 1 - 2 (June 24th, 25th):}
\label{sec:orgd981096}
\subsubsection{Equipment:}
\label{sec:org254c6b5}
Needles, Tennis balls, Measuring stick
Vernier caliper, protractor
Weighing machines,

\subsubsection{Brief: Introduction to physics}
\label{sec:org72e27c4}

Introduced scientific method and its application to physics. We touched briefly
about scope of physics and its day to day relevance. We touched upon
fundamentals of mathematics required for physics (the language of
physics/science).
Children learned about measuring different quantities like length, weight,
volume. Fundamental units (length, time, weight) as well as derived units
(density). were introduced.

\subsubsection{Transscript:}
\label{sec:org225d77a}

What is physics?
Physics is a branch of science,
Its roots are in curiocity of the man to  find out answer
to the question. The question of The life, the universe and everything in it.

Scientific method.

\begin{enumerate}
\item Make a hypotheses, based on observed data.
\item Find the limitations.
\item Create an experiment to verify the hypotheses.
\item If the experiment succeeds, confidence on the theory increases.
\item If the experiment fails???? hypotheses MUST BE WRONG or at least missing
something.
\end{enumerate}


In physics we deal with
Daily objects (Juggle tennis balls)

To absolute large, the end of the universe

And to the beginning and end of the TIME.

To absolute small,
(show the needle and ask to look at the pointed tip)

\[ \mbox{ Size of a needle point? : }
10^{-3}m\]
\[ \mbox{ Size of hydrogen atom : } \[ 0.5 * 10^{-10}m \]
\[ \mbox{ Size of carbon atom: }  1.54 * 10^{-10}m\]

This gives us
\[ 2 * 10^7 \mbox{ Hydrogen atoms or } \]

\[ 6.66 * 10^6 \mbox{ Carbon atoms } \]


How do we measure something?

Units

Recall Maitreyi's stick

\subsubsection{Fundamental Units:}
\label{sec:orgfca1904}

Length : \[m\]
Mass   : \[kg\]
Time   : \[sec\]

\subsubsection{Derived units}
\label{sec:orgb51896b}
Volume : \[m^3\]

Density : \[\frac{kg}{m^3}\]

\subsection{Lecture 3 - 4 : July week 1 (1st  and 2nd)}
\label{sec:org9ce94f1}

\subsubsection{Brief: Accuracy of measurements and measuring time}
\label{sec:org1aaf151}
We discussed multiple ways of increasing accuracy of measurements.
We discussed how to measure time. Direct measurements involve clocks.
Principles of clocks (pendulum).
Solved problems regarding pendulum and discussed properties of pendulum.

\subsubsection{Instruments}
\label{sec:orgf5fe4bc}
\begin{enumerate}
\item Accuracy of instruments
\item Increasing accuracy
\item Measurement of time
\end{enumerate}

\subsubsection{Direct measurements:}
\label{sec:org07e38be}
Measuring interval of pendulum

\subsubsection{Indirect measurement:}
\label{sec:orgd7c77e0}
Measuring thickness of a paper.
Measuring diameter of a sub mm tube.

\subsubsection{Homework:}
\label{sec:org2437ef3}
\begin{enumerate}
\item Read chapter 1 and take notes in rough book of main points. and solve in chapter problems for chapter 1.
\item Solve in chapter problems in rough book (Not end of chatper problems).
\end{enumerate}
\end{document}
